\documentclass[11pt,a4paper]{article}

% Packages
\usepackage[utf8]{inputenc}
\usepackage[T1]{fontenc}
\usepackage[margin=1.2in]{geometry}
\usepackage{amsmath,amsthm,amssymb}
\usepackage{mathtools}
\usepackage{enumitem}
\usepackage{hyperref}
\usepackage{cleveref}
\usepackage{thmtools}

% Hyperref setup
\hypersetup{
    colorlinks=true,
    linkcolor=blue,
    citecolor=blue,
    urlcolor=blue
}

% Theorem environments
\theoremstyle{plain}
\newtheorem{theorem}{Theorem}
\newtheorem{lemma}[theorem]{Lemma}
\newtheorem{proposition}[theorem]{Proposition}
\newtheorem{corollary}[theorem]{Corollary}

\theoremstyle{definition}
\newtheorem{definition}[theorem]{Definition}
\newtheorem{remark}[theorem]{Remark}
\newtheorem{example}[theorem]{Example}

% Custom commands
\newcommand{\R}{\mathbb{R}}
\newcommand{\C}{\mathbb{C}}
\newcommand{\N}{\mathbb{N}}
\newcommand{\Z}{\mathbb{Z}}
\newcommand{\D}{\mathbb{D}}
\newcommand{\T}{\mathbb{T}}
\newcommand{\abs}[1]{\left|#1\right|}
\newcommand{\norm}[1]{\left\|#1\right\|}
\DeclareMathOperator{\supp}{supp}
\renewcommand{\Re}{\operatorname{Re}}
\renewcommand{\Im}{\operatorname{Im}}

% Title information
\title{\textbf{Proof of the Herglotz Representation Theorem}}

\begin{document}
\date{}
\maketitle
\setlength{\parindent}{0pt}

\section{Preliminary Results}

Let $\D = \{z \in \C : \abs{z} < 1\}$ denote the unit disc and $\mathbb{H} = \{z \in \C : \Re z > 0\}$ the right half-plane. We consider the Borel $\sigma$-algebra on~$\C$.\\

We follow a proof that can be found, e.g., in \cite[Section 3]{Garnett1981}.\\

Let $C(\partial\D)$ be the space of continuous, real-valued functions on~$\partial\D$, equipped with the supremum norm $\norm{\cdot}_\infty$, and let $C(\partial\D)^*$ be its dual space.\\


We first need some technical lemmas that establish the compactness and convergence properties needed for the main theorem. 

\begin{lemma}[Compactness in the dual ball]\label{lem:compact}
The set 
\[
K = \{\Lambda \in C(\partial\D)^* : \abs{\Lambda(f)} \leq 1 \text{ for all } f \in C(\partial\D) \text{ with } \norm{f}_\infty \leq 1\}
\]
is a compact metric space with respect to the induced weak*-topology of $C(\partial\D)^*$.
\end{lemma}

\begin{proof}
By the Banach-Alaoglu theorem, $K$ is weak*-compact as the closed unit ball in the dual space.

Since $C(\partial\D)$ is a separable vector space (the unit circle is a compact metric space) and $K$ is weak*-compact, the space $K$ is metrizable in the weak*-topology. This follows because $C(\partial\D)^*$ has a countable family of continuous functions that separates points, given by evaluation at a countable dense subset of~$C(\partial\D)$.
\end{proof}

Let $p\colon \D \to \mathbb{H}$ be analytic with $p(0) = 1$. Let $u = \Re p$ denote its real part. 

Consider a sequence $(r_n)_{n\in\N}$ in $(0,1)$ such that $r_n \to 1$ as $n \to \infty$ (for example, $r_n = \frac{n+1}{n+2}$). For every $n \in \N$, define $u_n\colon \D \to \R$ by
\[
u_n(z) = u(r_n z), \quad z \in \D.
\]

For every $n \in \N$, define the linear functional $\Lambda_n \in C(\partial\D)^*$ by
\[
\Lambda_n(f) = \frac{1}{2\pi} \int_0^{2\pi} f(e^{it}) u_n(e^{it}) \, dt, \quad f \in C(\partial\D).
\]

\begin{lemma}[Boundedness of the functionals]\label{lem:bounded}
For every $n \in \N$, we have $\abs{\Lambda_n(f)} \leq 1$ for all $f \in C(\partial\D)$ with $\norm{f}_\infty \leq 1$.
\end{lemma}

\begin{proof}
Let $n \in \N$ and $f \in C(\partial\D)$ with $\norm{f}_\infty \leq 1$. Since $u_n(\zeta) > 0$ for all $\zeta \in \partial\D$ (as $u_n$ is the real part of a function mapping to the right half-plane), we have
\[
\abs{\Lambda_n(f)} \leq \frac{1}{2\pi} \int_0^{2\pi} \abs{f(e^{it})} u_n(e^{it}) \, dt \leq \frac{1}{2\pi} \int_0^{2\pi} u_n(e^{it}) \, dt.
\]

Since $u_n$ is harmonic on $\D$ and continuous on $\overline{\D}$, by the mean value property for harmonic functions,
\[
\frac{1}{2\pi} \int_0^{2\pi} u_n(e^{it}) \, dt = u_n(0) = u(0) = \Re p(0) = 1.
\]
This completes the proof.
\end{proof}

\begin{lemma}[Existence of a convergent subsequence]\label{lem:subsequence}
There exists a subsequence $(\Lambda_{n_k})_{k\in\N}$ of $(\Lambda_n)_{n\in\N}$ and a functional $\Lambda \in C(\partial\D)^*$ such that
\[
\lim_{k\to\infty} \Lambda_{n_k}(f) = \Lambda(f), \quad \text{for all } f \in C(\partial\D).
\]
\end{lemma}

\begin{proof}
By \Cref{lem:bounded}, the sequence $(\Lambda_n)_{n\in\N}$ lies in $K$. By \Cref{lem:compact}, there exist $\Lambda \in K$ and a subsequence $(\Lambda_{n_k})_{k\in\N}$ of $(\Lambda_n)_{n\in\N}$ that is weak*-convergent to~$\Lambda$.
\end{proof}

\section{The Representing Measure}

\begin{lemma}[Existence of the probability measure]\label{lem:measure}
There exists a probability measure $\mu$ on $\partial\D$ and a subsequence $(r_{n_k})_{k\in\N}$ such that
\[
\lim_{k\to\infty} \frac{1}{2\pi} \int_0^{2\pi} f(e^{it}) u(r_{n_k} e^{it}) \, dt = \int_{\partial\D} f(\zeta) \, d\mu(\zeta)
\]
for all $f \in C(\partial\D)$ with $f(\zeta) \geq 0$ for all $\zeta \in \partial\D$.
\end{lemma}

\begin{proof}
Let $\Lambda \in C(\partial\D)^*$ and $(r_{n_k})_{k\in\N}$ be given by \Cref{lem:subsequence}. Since $u(z) > 0$ for all $z \in \D$ (as $p$ maps to the right half-plane), in view of \Cref{lem:subsequence}, $\Lambda$ is a positive functional on~$C(\partial\D)$. 

By the Riesz-Markov-Kakutani representation theorem, there exists a measure $\mu$ on $\partial\D$ such that 
\[
\Lambda(f) = \int_{\partial\D} f(\zeta) \, d\mu(\zeta)
\]
for all $f \in C(\partial\D)$ with $f(\zeta) \geq 0$ for all $\zeta \in \partial\D$. 

For $f \equiv 1$, by the mean value property, we have
\[
\mu(\partial\D) = \lim_{k\to\infty} \frac{1}{2\pi} \int_0^{2\pi} u(r_{n_k} e^{it}) \, dt = 1.
\]
Therefore, $\mu$ is a probability measure.
\end{proof}

\begin{lemma}[Properties of the Herglotz kernel integral]\label{lem:herglotz_kernel}
Let $\mu$ be a probability measure on $\partial\D$ and define
\[
q(z) = \int_{\partial\D} \frac{\zeta + z}{\zeta - z} \, d\mu(\zeta), \quad z \in \D.
\]
Then $q$ is analytic on~$\D$, takes values in~$\mathbb{H}$, and satisfies $q(0) = 1$.
\end{lemma}

\begin{proof}
The function $z \mapsto \frac{\zeta + z}{\zeta - z}$ is analytic for $z \in \D$ and $\zeta \in \partial\D$ (since $\abs{z} < 1 = \abs{\zeta}$). By dominated convergence and Morera's theorem, $q$ is analytic on~$\D$.

For the real part, note that
\[
\Re \frac{\zeta + z}{\zeta - z} = \frac{1 - \abs{z}^2}{\abs{\zeta - z}^2} > 0
\]
for $z \in \D$ and $\zeta \in \partial\D$. Therefore, $q$ maps to~$\mathbb{H}$.

Finally, $q(0) = \int_{\partial\D} \frac{\zeta}{\zeta} \, d\mu(\zeta) = \mu(\partial\D) = 1$.
\end{proof}

\section{Key Technical Lemmas}

\begin{lemma}[Poisson-type integral formula]\label{lem:poisson}
For every $r \in (0,1)$,
\[
u(rz) = \frac{1}{2\pi} \int_0^{2\pi} \Re \frac{e^{it} + z}{e^{it} - z} \cdot u(re^{it}) \, dt, \quad \text{for all } z \in \D.
\]
\end{lemma}

\begin{proof}
Fix $r \in (0,1)$ and $z \in \D$. The function $p(r\,\cdot)$ is analytic on $\frac{1}{r}\D$, and since $\frac{1}{r} > 1$, by the Cauchy integral formula,
\[
p(rz) = \frac{1}{2\pi i} \int_{\partial\D} \frac{p(r\zeta)}{\zeta - z} \, d\zeta = \frac{1}{2\pi} \int_0^{2\pi} \frac{p(re^{it}) e^{it}}{e^{it} - z} \, dt.
\]

Since $\abs{\frac{1}{\overline{z}}} > 1$, the function $\zeta \mapsto \frac{p(r\zeta)}{\frac{1}{\overline{z}} - \zeta}$ is analytic on a disc centered at~$0$ with radius larger than~$1$. By the Cauchy-Goursat theorem for a circle,
\[
0 = \frac{1}{2\pi i} \int_{\partial\D} \frac{p(r\zeta)}{\frac{1}{\overline{z}} - \zeta} \, d\zeta = \frac{1}{2\pi} \int_0^{2\pi} \frac{p(re^{it}) \overline{z}}{e^{-it} - \overline{z}} \, dt.
\]

Adding these two integral formulas yields
\[
p(rz) = \frac{1}{2\pi} \int_0^{2\pi} p(re^{it}) \cdot \frac{1 - \abs{z}^2}{\abs{e^{it} - z}^2} \, dt = \frac{1}{2\pi} \int_0^{2\pi} p(re^{it}) \cdot \Re \frac{e^{it} + z}{e^{it} - z} \, dt.
\]

Taking the real part of both sides completes the proof.
\end{proof}

\begin{lemma}[Uniqueness via real parts]\label{lem:uniqueness_real}
Let $p, q\colon \D \to \mathbb{H}$ be analytic with $\Re p = \Re q$ and $p(0) = q(0)$. Then $p = q$.
\end{lemma}

\begin{proof}
The function $p - q$ is analytic on~$\D$ and $\Re(p - q) \equiv 0$ on~$\D$. By the Cauchy-Riemann equations, $(p - q)' \equiv 0$. Therefore, $p - q$ is constant on~$\D$, and since $p(0) = q(0)$, we conclude $p = q$.
\end{proof}

\section{Main Theorem}

\begin{theorem}[Herglotz representation theorem]\label{thm:herglotz}
Let $p\colon \D \to \mathbb{H}$ be analytic with $p(0) = 1$. Then there exists a \emph{unique} probability measure $\mu$ on $\C$ with support on~$\partial\D$ such that
\[
p(z) = \int_{\partial\D} \frac{\zeta + z}{\zeta - z} \, d\mu(\zeta), \quad \text{for all } z \in \D.
\]
\end{theorem}

\begin{proof}
\textbf{Existence.} Let $\mu$ be given by \Cref{lem:measure}. Let $z \in \D$. Using \Cref{lem:measure,lem:herglotz_kernel,lem:poisson}, we have
\begin{align*}
\int_{\partial\D} \Re \frac{e^{it} + z}{e^{it} - z} \, d\mu(\zeta) 
&= \lim_{k\to\infty} \frac{1}{2\pi} \int_0^{2\pi} \Re \frac{e^{it} + z}{e^{it} - z} \cdot u(r_{n_k} e^{it}) \, dt \\
&= \lim_{k\to\infty} u(r_{n_k} z) = u(z).
\end{align*}

Let $q$ be as in \Cref{lem:herglotz_kernel}. Then $\Re q = \Re p = u$ and $q(0) = p(0) = 1$. By \Cref{lem:uniqueness_real}, $p = q$, so $p$ has the desired representation.

\medskip
\noindent\textbf{Uniqueness.} Assume there exist probability measures $\mu_1, \mu_2$ on $\C$ with support on~$\partial\D$ such that
\[
p(z) = \int_{\partial\D} \frac{\zeta + z}{\zeta - z} \, d\mu_1(\zeta) = \int_{\partial\D} \frac{\zeta + z}{\zeta - z} \, d\mu_2(\zeta), \quad z \in \D.
\]

Let $\eta = \mu_2 - \mu_1$. Then $\eta$ is a signed finite measure on $\C$ with support on~$\partial\D$. Moreover, for every $j \in \{1, 2\}$, using the geometric series expansion (valid for $\abs{z} < 1$),
\[
\int_{\partial\D} \frac{\zeta + z}{\zeta - z} \, d\mu_j(\zeta) = \int_{\partial\D} \left(1 + 2\sum_{n=1}^\infty \zeta^{-n} z^n\right) d\mu_j(\zeta) = \mu_j(\partial\D) + 2\sum_{n=1}^\infty z^n \int_{\partial\D} \zeta^{-n} \, d\mu_j(\zeta).
\]

Therefore,
\[
\int_{\partial\D} \zeta^{-n} \, d\mu_1(\zeta) = \int_{\partial\D} \zeta^{-n} \, d\mu_2(\zeta), \quad \text{for all } n \in \N.
\]

This implies
\[
\int_{\partial\D} f(\zeta) \, d\mu_1(\zeta) = \int_{\partial\D} f(\zeta) \, d\mu_2(\zeta)
\]
for every polynomial $f$ on~$\C$. Since the polynomials (or equivalently, the Laurent polynomials $\{\zeta^n : n \in \Z\}$) are dense in $C(\partial\D)$ by the Stone-Weierstrass theorem, and both $\mu_1$ and $\mu_2$ are probability measures, we conclude $\mu_1 = \mu_2$.
\end{proof}

\begin{remark}
The Herglotz representation theorem is fundamental in complex analysis and harmonic analysis. It provides a correspondence between analytic functions with positive real part and probability measures on the circle, analogous to how the Poisson integral formula represents harmonic functions in terms of their boundary values.
\end{remark}

\begin{remark}
The kernel $\frac{\zeta + z}{\zeta - z}$ appearing in the theorem is known as the \emph{Herglotz kernel} or \emph{Riesz-Herglotz kernel}. Its real part,
\[
\Re \frac{\zeta + z}{\zeta - z} = \frac{1 - \abs{z}^2}{\abs{\zeta - z}^2},
\]
is the Poisson kernel for the unit disc.
\end{remark}

\begin{thebibliography}{9}

\bibitem{Garnett1981}
John B.~Garnett, \emph{Bounded Analytic Functions}, Section~3, Saunders College Publishing/Harcourt Brace, 1981.

\bibitem{RudinRCA1987}
Walter~Rudin, \emph{Real and Complex Analysis}, 3rd edition, McGraw-Hill, 1987.

\bibitem{RudinFA1991}
Walter~Rudin, \emph{Functional Analysis}, 2nd edition, McGraw-Hill, 1991.

\end{thebibliography}

\end{document}