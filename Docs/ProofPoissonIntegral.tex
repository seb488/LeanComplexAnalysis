\documentclass[11pt,a4paper]{article}

% Packages
\usepackage[utf8]{inputenc}
\usepackage[T1]{fontenc}
\usepackage[margin=1.2in]{geometry}
\usepackage{amsmath,amsthm,amssymb}
\usepackage{mathtools}
\usepackage{enumitem}
\usepackage{hyperref}
\usepackage{cleveref}
\usepackage{thmtools}
\usepackage{graphicx}

% Hyperref setup
\hypersetup{
    colorlinks=true,
    linkcolor=blue,
    citecolor=blue,
    urlcolor=blue
}

% Theorem environments
\theoremstyle{plain}
\newtheorem{theorem}{Theorem}
\newtheorem{lemma}[theorem]{Lemma}
\newtheorem{proposition}[theorem]{Proposition}
\newtheorem{corollary}[theorem]{Corollary}

\theoremstyle{definition}
\newtheorem{definition}[theorem]{Definition}
\newtheorem{remark}[theorem]{Remark}
\newtheorem{example}[theorem]{Example}

% Custom commands
\newcommand{\R}{\mathbb{R}}
\newcommand{\C}{\mathbb{C}}
\newcommand{\N}{\mathbb{N}}
\newcommand{\Z}{\mathbb{Z}}
\newcommand{\D}{\mathbb{D}}
\newcommand{\T}{\mathbb{T}}
\newcommand{\abs}[1]{\left|#1\right|}
\newcommand{\norm}[1]{\left\|#1\right\|}
\newcommand{\set}[1]{\left\{#1\right\}}
\DeclareMathOperator{\supp}{supp}
\renewcommand{\Re}{\operatorname{Re}}
\renewcommand{\Im}{\operatorname{Im}}

% Title information
\title{\textbf{Proof of the Poisson Integral Formula}}
\author{}
\date{}

\begin{document}

\maketitle

\setlength{\parindent}{0pt}
\setlength{\parskip}{0.5em}



\section{Main Results}


\begin{theorem}[Poisson Integral Formula]
\label{thm:poisson}
Let $u : \C \to \R$ be harmonic on the open unit disc $\D = \set{z \in \C : \abs{z} < 1}$ and continuous on the closed unit disc $\overline{\D}$. Then for all $z \in \D$,
\[
u(z) = \frac{1}{2\pi} \int_0^{2\pi} \frac{1 - \abs{z}^2}{\abs{e^{it} - z}^2} \, u(e^{it}) \, dt.
\]
\end{theorem}

The kernel 
\[
P(z, e^{it}) = \frac{1 - \abs{z}^2}{\abs{e^{it} - z}^2}
\]
is called the \emph{Poisson kernel} for the unit disc.

\
\section{Key Lemmas and Intermediate Results}

Let $E$ be (e.g.) a complex Banach space.

\begin{lemma}[Cauchy Formula on Scaled Discs]
\label{lem:cauchy_scaled}
Let $f : \C \to E$ be analytic on $\D$, and let $z \in \D$, $r \in (0,1)$. Then
\[
f(rz) = \frac{1}{2\pi} \int_0^{2\pi} \frac{e^{it}}{e^{it} - z} \, f(re^{it}) \, dt.
\]
\end{lemma}

This follows from the standard Cauchy integral formula applied to the circle of radius $r$.


\begin{lemma}[Goursat Vanishing Integral]
\label{lem:goursat}
Under the same hypotheses as \cref{lem:cauchy_scaled}, we have
\[
\int_0^{2\pi} \frac{\overline{z}}{\overline{e^{it}} - \overline{z}} \, f(re^{it}) \, dt = 0.
\]
\end{lemma}

This is a consequence of the Cauchy--Goursat theorem.


\begin{lemma}[Real Part of Herglotz Kernel]
\label{lem:kernel_connection}
Let $x,w\in\C$. For $\abs{x} = 1$ and $w \in \D$,
\[
\Re\left(\frac{x + w}{x - w}\right) = \frac{1 - \abs{w}^2}{\abs{x - w}^2}.
\]
\end{lemma}

This identity connects the Herglotz kernel to the Poisson kernel and is verified by direct algebraic computation.

\section{Poisson Formula for Analytic Functions}

\begin{proposition}[Poisson Formula for Analytic Functions]
\label{prop:poisson_analytic}
Let $f : \C \to E$ be analytic on $\D$, and let $z \in \D$, $r \in (0,1)$. Then
\[
f(rz) = \frac{1}{2\pi} \int_0^{2\pi} \Re\left(\frac{e^{it} + z}{e^{it} - z}\right) \, f(re^{it}) \, dt.
\]
\end{proposition}

This follows by combining \cref{lem:cauchy_scaled,lem:goursat,lem:kernel_connection}.

\section{Proof of the Poisson Integral Formula}

The proof of \cref{thm:poisson} proceeds in several steps:

\begin{enumerate}[label=(\roman*)]
\item \textbf{Harmonic extension:} Since $u$ is harmonic on $\D$, there exists an analytic function $f : \D \to \C$ such that $u = \Re f$ on $\D$.

\item \textbf{Scaled formula:} For $r \in (0,1)$ and $z \in \D$, apply \cref{prop:poisson_analytic} to obtain
\[
u(rz) = \frac{1}{2\pi} \int_0^{2\pi} \Re\left(\frac{e^{it} + z}{e^{it} - z}\right) \, u(re^{it}) \, dt.
\]

\item \textbf{Approximation sequence:} Consider a sequence $r_n \to 1$ with $r_n \in (0,1)$ and the integral representation of $u(r_n \cdot)$. For $r_n \to 1$ we can apply the dominated convergence theorem as the integrand is uniformly bounded (using continuity of $u$ on $\overline{\D}$ and the fact that the Poisson kernel is bounded for $z$ in a compact subset of $\D$).\\

The continuity of $u$ on $\overline{\D}$ ensures that $u(r_n e^{it}) \to u(e^{it})$ as $r_n \to 1$, yielding the desired formula.
\end{enumerate}


\section{A Counterexample}

We present an example showing that a harmonic function on $\D$ can be extended continuously to $\overline{\D}$, while this is not possible for the associated holomorphic function $f$.

\begin{example}[Failure of Poisson Formula without Continuity]
\label{ex:counterexample}
Consider the function defined by the power series
\[
f(z) = -i \sum_{n=2}^{\infty} \frac{z^n}{n \log n}.
\]
This series has radius of convergence $R = 1$ because
\[
\lim_{n \to \infty} \frac{\abs{a_n}}{\abs{a_{n+1}}} = \lim_{n \to \infty} \frac{n+1}{n} \cdot \frac{\log(n+1)}{\log n} = 1.
\]
Thus, $f$ is holomorphic on the open unit disc $\D$.

\textbf{Analysis of the Real Part:} Let $z = e^{i\theta}$ be a point on the boundary $\partial \D$. The real part of $f(z)$ on the boundary is
\[
\Re f(e^{i\theta}) = \Re\left(-i \sum_{n=2}^{\infty} \frac{\cos(n\theta) + i\sin(n\theta)}{n \log n}\right) = \sum_{n=2}^{\infty} \frac{\sin(n\theta)}{n \log n}.
\]
This is a trigonometric sine series with coefficients $b_n = \frac{1}{n \log n}$. The coefficients satisfy:
\begin{enumerate}
\item They are monotonically decreasing to zero: $b_n \searrow 0$.
\item The sequence $n b_n = \frac{1}{\log n}$ tends to $0$ as $n \to \infty$.
\end{enumerate}

By the Chaundy--Jolliffe Theorem (or standard results on uniform convergence of sine series), if $(b_n)$ is monotonic decreasing and $n b_n \to 0$, then the series $\sum b_n \sin(n\theta)$ converges uniformly on $\theta \in [0, 2\pi]$.

Since the series converges uniformly to a continuous function on the circle $\partial \D$, and $\Re f$ is the harmonic extension of this boundary function to the interior, $\Re f$ is continuous on the closed unit disc $\overline{\D}$.

\textbf{Analysis of the Imaginary Part:} The imaginary part of $f(z)$ on the boundary would correspond to the series
\[
\Im f(e^{i\theta}) = \Im\left(-i \sum_{n=2}^{\infty} \frac{\cos(n\theta) + i\sin(n\theta)}{n \log n}\right) = -\sum_{n=2}^{\infty} \frac{\cos(n\theta)}{n \log n}.
\]

Consider the behavior at $\theta = 0$ (i.e., $z = 1$). The series becomes
\[
-\sum_{n=2}^{\infty} \frac{1}{n \log n}.
\]
By the integral test,
\[
\int_2^{\infty} \frac{dx}{x \log x} = \left[\log(\log x)\right]_2^{\infty} = \infty.
\]
Thus, the series diverges to $-\infty$ at $z = 1$.

\textbf{Conclusion:} Inside the disc, as $r \to 1^-$ along the real axis, the value of 
\[
\Im f(r) = -\sum_{n=2}^{\infty} \frac{r^n}{n \log n}
\]
tends to $-\infty$. The function $f$ is unbounded near $z = 1$, so it cannot be extended to a continuous function on the closed unit disc $\overline{\D}$.

\end{example}

\begin{thebibliography}{99}

\bibitem{rudin2006real}
W. Rudin, \emph{Real and Complex Analysis}, 3rd ed., McGraw-Hill, 1987.

\end{thebibliography}

\end{document}